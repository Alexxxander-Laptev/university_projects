\documentclass[14pt, oneside]{altsu-bachelor}

\title{Сервис автоматизированного решения CAPTCHA}
\author{А.\,В.~Лаптев}
\groupnumber{5.306M}
\GradebookNumber{1337}
\supervisor{В.\,В.~Электроник}
\supervisordegree{к.ф.-м.н., доцент}
\ministry{Министерство науки и высшего образования}
\country{Российской Федерации}
\fulluniversityname{ФГБОУ ВО Алтайский государственный университет}
\institute{Институт цифровых технологий, электроники и физики}
\department{Кафедра вычислительной техники и электроники}
\departmentchief{В.\,В.~Пашнев}
\departmentchiefdegree{к.ф.-м.н., доцент}
\shortdepartment{ВТиЭ}
\ChairmanOfTheStateCertificationCommission{С.\,П.~Пронин}
\ChairmanOfTheStateCertificationCommissiondegree{д.т.н., проф.}
\NormController{А.\,В.~Калачёв}
\NormControllerdegree{к.ф.-м.н., доцент}
\Consultant{}
\Consultantdegree{}
\UDC{004.94}
\docname{БР 09.03.01}
\abstractRU{Объём текста не менее 1000 символов! Пока счётчики выставляются в ручную, при необходимости правьте cls-файл.}
\abstractEN{Большой текст на английском!}
\keysRU{компьютерное моделирование, cистема управления версиями}
\keysEN{computer simulation, distributed version control}
\countWorkPage{22}
\countWorkImg{6}
\countWorkLit{5}
\countWorkTab{6}

\date{\the\year}

% Подключение файлов с библиотекой.
\addbibresource{graduate-students.bib}

\begin{document}
\maketitle

\setcounter{page}{2}
\makeabstract
\tableofcontents

\chapter*{ВВЕДЕНИЕ}
\addcontentsline{toc}{chapter}{ВВЕДЕНИЕ}

\textbf{Актуальность}

\textbf{Цель}

\textbf{Задачи}
\begin{enumerate}
\item Текст много текста, очень много текста. Текст много текста, очень много текста.
\item Текст много текста, очень много текста. Текст много текста, очень много текста. Текст много текста, очень много текста.
\item Текст много текста, очень много текста. Текст много текста, очень много текста. Текст много текста, очень много текста. Текст много текста, очень много текста. Текст много текста, очень много текста. Текст много текста, очень много текста.
\end{enumerate}

% \input{chapter-1-csae}
% \input{chapter-2-csae}
% \input{chapter-3-csae}

\chapter*{ЗАКЛЮЧЕНИЕ}
\addcontentsline{toc}{chapter}{ЗАКЛЮЧЕНИЕ}

\begin{enumerate}
\item Пример ссылки на литературу~\cite{bookexample}.
\item Пример ссылки на литературу~\cite{book1author}.
\item Пример ссылки на литературу~\cite{book5author}.
\item Пример ссылки на литературу~\cite{wikiRUBitbucket}.
\item Пример ссылки на литературу~\cite{wikiRUGitHub}.
\item Пример ссылки на литературу~\cite{wikiRUIdSoftware}.
\end{enumerate}

Всего рисунков в документе~\totalfigures.
\newpage
\addcontentsline{toc}{chapter}{СПИСОК ИСПОЛЬЗОВАННОЙ ЛИТЕРАТУРЫ}
\printbibliography[title={СПИСОК ИСПОЛЬЗОВАННОЙ ЛИТЕРАТУРЫ}]

\appendix
\newpage
\chapter*{\raggedleft\label{appendix1}Приложение}
\phantomsection\addcontentsline{toc}{chapter}{ПРИЛОЖЕНИЕ}

\begin{center}
\label{code:appendix}Текст программы
\end{center}

\begin{code}
    \captionof{listing}{\label{code:audiocaptcha}Исходный код расшифровки Audio CAPTCHA}
    \vspace{-1cm}
    \inputminted{python}{code/audiocaptcha/audiocaptcha.py}
\end{code}

\begin{code}
    \captionof{listing}{\label{code:audiocaptcha-solve}Исходный код автоматизированного решения Audio CAPTCHA}
    \vspace{-1cm}
    \inputminted{python}{code/audiocaptcha/audiocaptcha_solve.py}
\end{code}

\begin{code}
    \captionof{listing}{\label{code:gen-dataset}Исходный код генератора синтетических CAPTCHA}
    \vspace{-1cm}
    \inputminted{python}{code/textcaptcha/gen-dataset.py}
\end{code}

\begin{code}
    \captionof{listing}{\label{code:preprocessing}Исходный код для предобработки изображений датасета}
    \vspace{-1cm}
    \inputminted{python}{code/textcaptcha/preprocessing.py}
\end{code}

\begin{code}
    \captionof{listing}{\label{code:tf-dataset}Исходный код для создания датасета в формате тензоров}
    \vspace{-1cm}
    \inputminted{python}{code/textcaptcha/tf-dataset.py}
\end{code}

\begin{code}
    \captionof{listing}{\label{code:crnn}Исходный код CRNN модели}
    \vspace{-1cm}
    \inputminted{python}{code/textcaptcha/crnn.py}
\end{code}

\begin{code}
    \captionof{listing}{\label{code:seq2seq}Исходный код Seq-to-Seq модели}
    \vspace{-1cm}
    \inputminted{python}{code/textcaptcha/seq-to-seq.py}
\end{code}

\begin{code}
    \captionof{listing}{\label{code:test-model}Исходный код тестирования модели}
    \vspace{-1cm}
    \inputminted{python}{code/textcaptcha/test.py}
\end{code}

\begin{code}
    \captionof{listing}{\label{code:get-captcha}Исходный код получения CAPTCHA с целевого сайта}
    \vspace{-1cm}
    \inputminted{python}{code/imagecaptcha/get_captcha.py}
\end{code}

\begin{code}
    \captionof{listing}{\label{code:recognize}Исходный код дообучения модели на датасете}
    \vspace{-1cm}
    \inputminted{python}{code/imagecaptcha/recognize_objects.py}
\end{code}

\begin{code}
    \captionof{listing}{\label{code:solve-captcha}Исходный код автоматизированного решения CAPTCHA}
    \vspace{-1cm}
    \inputminted{python}{code/imagecaptcha/solve_captcha.py}
\end{code}

\makelastpage
\end{document}
