\chapter{Обзор технологии CAPTCHA}

\section{Краткая история и назначение CAPTCHA}

Проверочный код CAPTCHA (Completely Automated Public Turing test to tell Computers 
and Humans Apart) -- это метод защиты, основанный на принципе аутентификации 
<<вызов-ответ>>. Он предназначен для предотвращения автоматических действий, таких 
как спам или попытки взлома учетных записей, путем выполнения пользователем 
простого теста, подтверждающего, что он человек, а не программа [1]. Термин был 
придуман в 2003 году (https://link.springer.com/chapter/10.1007/3-540-39200-9\_18).

Исторически распространенный тип CAPTCHA был впервые изобретен в 1997 году двумя 
группами, работающими параллельно. Эта форма CAPTCHA требует ввода 
последовательности букв или цифр из искаженного изображения. Поскольку тест 
проводится компьютером, в отличие от стандартного теста Тьюринга, который 
проводится человеком, CAPTCHA иногда описываются как обратные тесты Тьюринга 
(https://isyou.info/jowua/papers/jowua-v4n3-3.pdf).

Набравшая популярность технология reCAPTCHA, была приобретена Google в 2009 году 
(https://googleblog.blogspot.com/2009/09/teaching-computers-to-read-google.html). 
В дополнение к предотвращению мошенничества с ботами для пользователей, Google 
использовал технологию reCAPTCHA для оцифровки архивов The New York Times и книг 
из Google Books в 2011 году (https://www.nytimes.com/2011/03/29/science/29recaptcha.html).

На сегодняшний день CAPTCHA является важной мерой безопасности, так как 
предотвращает автоматические атаки, например, массовую регистрацию ботов, и 
защищает данные пользователя. Современные системы CAPTCHA используют не только 
текст, но и изображения, аудио, поведенческие анализы и другие инновационные 
подходы, чтобы сделать тесты удобными для людей, но сложными для программ. 
Среднестатистическому человеку требуется около 10 секунд, чтобы решить типичный 
CAPTCHA.

\section{Классификация CAPTCHA по формату взаимодействия с пользователем}

На сегодняшний день наиболее распространенные виды CAPTCHA включают:

\begin{enumerate}
    \item reCAPTCHA -- разработанная Google система, которая предлагает тесты 
    на основе распознавания объектов, анализа поведения или текстовых символов.
    \item hCAPTCHA -- альтернатива reCAPTCHA, фокусирующаяся на защите 
    конфиденциальности пользователей.
    \item Capy -- система CAPTCHA, предлагающая пользователю головоломки, 
    например, сборку изображения или взаимодействие с элементами интерфейса [2].
\end{enumerate}

reCAPTCHA -- система защиты от автоматизированных действий, разработанная Google, 
которая помогает различать человека и бота. Она объединяет несколько подходов, 
делая проверку удобной для пользователей, но сложной для автоматических систем 
[3].

reCAPTCHA включает в себя следующие версии
~(https://en.wikipedia.org/wiki/ReCAPTCHA):

\begin{enumerate}
    \item reCAPTCHA v1 (устарела в 2018 году):
    \begin{enumerate}
        \item пользователи вводили текст, состоящий из искаженных слов, 
        отображаемых на изображении;
        \item использовала слова из книг и документов, которые не могли быть 
        распознаны OCR.
    \end{enumerate}
    \item reCAPTCHA v2:
    \begin{enumerate}
        \item клик по флажку: пользователи подтверждают, что они не роботы, 
        нажимая на флажок «Я не робот»;
        \item выбор объектов на изображениях: пользователи идентифицируют 
        заданные объекты на сетке из картинок;
        \item аудио CAPTCHA: для пользователей с ограничениями зрения, 
        предлагается прослушать запись и ввести услышанные символы.
    \end{enumerate}
    \item reCAPTCHA v3:
    \begin{enumerate}
        \item полностью работает в фоновом режиме, анализируя поведение 
        пользователя на странице;
        \item не требует явных действий, если пользователь считается 
        низкорискованным [4].
    \end{enumerate}
\end{enumerate}

hCAPTCHA -- это альтернативная система CAPTCHA, разработанная для защиты сайтов 
от ботов и спама, при этом уделяющая особое внимание конфиденциальности 
пользователей. Она стала популярной благодаря своей гибкости и ориентации на 
защиту данных [5].  

Основные особенности hCAPTCHA:

\begin{enumerate}
    \item конфиденциальность:
    \begin{enumerate}
        \item в отличие от reCAPTCHA, hCAPTCHA не собирает данные о 
        пользователях для рекламных целей, что делает ее привлекательной с точки 
        зрения соблюдения конфиденциальности.
    \end{enumerate}
    \item простота интеграции:
    \begin{enumerate}
        \item легко интегрируется с web-сайтами через API;
        \item совместима с большинством популярных платформ, таких как WordPress, 
        и может быть настроена для разных типов взаимодействия.
    \end{enumerate}
    \item модели монетизации:
    \begin{enumerate}
        \item владельцы сайтов могут зарабатывать, разрешая hCAPTCHA 
        использовать проверочные задачи, связанные с машинным обучением, 
        например, разметку данных.
    \end{enumerate}
\end{enumerate}

Виды взаимодействия с пользователями:

\begin{enumerate}
    \item графическая CAPTCHA: выбор изображений, соответствующих запросу;
    \item текстовая CAPTCHA: ввод символов (редко используется);
    \item аудио CAPTCHA: для пользователей с ограниченными возможностями, 
    предлагается прослушать и ввести услышанные символы;
    \item клик CAPTCHA: нажатие на флажок «Я не робот» (для низкорискованных 
    пользователей).
\end{enumerate}

Capy CAPTCHA -- это инновационная система CAPTCHA, разработанная с акцентом на 
удобство для пользователей и адаптацию к современным web-средам. Она предлагает 
интерактивные методы проверки, направленные на минимизацию раздражения 
пользователей при сохранении высокого уровня защиты от ботов [6].

Основные особенности Capy CAPTCHA:

\begin{enumerate}
    \item интерактивность:
    \begin{enumerate}
        \item Capy использует методы проверки, которые требуют не просто ввода 
        текста или выбора картинок, а выполнения задач, таких как перемещение 
        объектов;
        \item простые задачи делают процесс проверки менее раздражающим и более 
        интуитивным;
    \end{enumerate}
    \item гибкость настройки:
    \begin{enumerate}
        \item система может быть адаптирована под конкретные нужды сайта, 
        включая выбор сложности задач и дизайн интерфейса.
    \end{enumerate}
    \item доступность:
    \begin{enumerate}
        \item подходит для пользователей с различными потребностями, включая 
        мобильные устройства.
    \end{enumerate}
\end{enumerate}

Виды взаимодействия с пользователями:

\begin{enumerate}
    \item головоломки (Puzzle CAPTCHA): сборка пазла с перемещением недостающих 
    элементов в нужное место;
    \item тесты на логику и распознавание: выбор нужного объекта или 
    логического варианта из предложенных;
    \item текстовая CAPTCHA (редко используется).
\end{enumerate}

Capy CAPTCHA используется на сайтах, где важны как защита от ботов, так и 
положительный пользовательский опыт. Особенно популярна в проектах с высоким 
акцентом на дизайн и пользовательское взаимодействие.

\section{Критерии надежности и уязвимости различных систем CAPTCHA}

Эффективность CAPTCHA-систем определяется совокупностью признаков. К основным 
критериям надежности относятся~(https://en.wikipedia.org/wiki/CAPTCHA):

\begin{enumerate}
    \item устойчивость к машинному распознаванию, в том числе с использованием 
    современных алгоритмов искусственного интеллекта;
    \item наличие разнообразных и уникальных тестов, исключающих возможность 
    формирования обучающих или атакующих датасетов;
    \item доступность и понятность графического пользовательского интерфейса для 
    широкой аудитории.
\end{enumerate}

Несмотря на свою популярность, CAPTCHA-системы обладают рядом уязвимостей, 
снижающих их надежность и ухудшающих пользовательский опыт
~(https://eitca.org/cybersecurity/eitc-is-wasf-web-applications-security-fundamentals/authentication-eitc-is-wasf-web-applications-security-fundamentals/webauthn/examination-review-webauthn/what-are-the-main-vulnerabilities-and-limitations-associated-with-traditional-text-based-captchas/,
https://www.imperva.com/learn/application-security/what-is-captcha/\#what-is-captcha):

\begin{enumerate}
    \item высокая когнитивная нагрузка, связанная со сложностью задач для 
    человека;
    \item недоступность или трудности прохождения для отдельных групп 
    пользователей, включая людей с нарушениями зрения или слуха;
    \item низкая эффективность против целевых атак или сервисов, управляемых 
    человеком;
    \item несовместимость с некоторыми web-браузерами и мобильными устройствами;
    \item ограниченная поддержка вспомогательных технологий, используемых людьми 
    с ограниченными возможностями.
\end{enumerate}

Для CAPTCHA в текстовом формате выделяют следующие критерии надежности:

\begin{enumerate}
    \item преднамеренное искажение символов (геометрическая деформация, 
    перекрытие, наклоны);
    \item использование нестандартных шрифтов, графических помех и шумов;
    \item отсутствие четкой сегментации между символами, затрудняющей их 
    раздельное распознавание;
    \item рандомизация длины строк и набора используемых символов.
\end{enumerate}

Несмотря на совокупность методов для усложнения автоматизированного 
распознавания, текстовые CAPTCHA подвержены следующим уязвимостям:

\begin{enumerate}
    \item современные OCR-системы и seq2seq-модели, в том числе архитектуры на 
    основе CNN и RNN, успешно справляются с распознаванием даже при наличии 
    искажений;
    \item упрощенные CAPTCHA без шумов и дополнительных помех могут быть 
    распознаны с высокой точностью даже базовыми алгоритмами;
    \item использование ограниченного и фиксированного алфавита позволяет обучать 
    модели, показывающие высокую точность при распознавании.
\end{enumerate}

Для CAPTCHA в аудиоформате основными характеристиками надежности являются:

\begin{enumerate}
    \item введение фонового шума и аудиоискажений, затрудняющих автоматическую 
    обработку;
    \item использование слов, сходных по звучанию, нестандартных акцентов и 
    синтезированной речи;
    \item наложение голосов, изменение темпа и интонации произношения;
    \item высокая вариативность аудиофайлов.
\end{enumerate}

В то же время современным реализациям аудио CAPTCHA присущи следующие уязвимости:

\begin{enumerate}
    \item преобразование аудио в спектрограммы с последующим анализом с помощью 
    CNN и методов CTC позволяет достигать высокой точности распознавания;
    \item современные модели автоматического распознавания речи успешно решают 
    даже зашумленные аудиозадания;
    \item применение генеративных моделей и других методов предварительной 
    обработки аудио позволяет эффективно устранять шумы, повышая точность 
    распознавания.
\end{enumerate}

К ключевым характеристикам надежности CAPTCHA с изображениями относятся:

\begin{enumerate}
    \item использование изображений из реального мира с вариативными фонами и 
    сценами;
    \item намеренное смещение объектов по положению, углу поворота и масштабу;
    \item включение визуально схожих ложных объектов, усложняющих выбор 
    правильных;
    \item разнообразие типов изображений.
\end{enumerate}

Среди основных уязвимостей графических CAPTCHA можно выделить:

\begin{enumerate}
    \item высокая эффективность современных моделей детектирования объектов 
    (например, YOLOv8, Faster R-CNN) при наличии специализированного обучающего 
    датасета;
    \item возможность автоматизации взаимодействия с CAPTCHA (например, выбор 
    изображений) с использованием скриптов и эмуляторов браузеров;
    \item ограниченность числа классов, используемых в задаче, позволяет быстро 
    обучить модель для решения конкретной CAPTCHA.
\end{enumerate}
