% Теоретическая глава про все виды CAPTCHA

\chapter{Современные методы защиты от ботов и спама на основе CAPTCHA}

\section{История CAPTCHA}

Проверочный код CAPTCHA -- это метод защиты, основанный на принципе 
аутентификации «вызов-ответ». Он предназначен для предотвращения автоматических 
действий, таких как спам или попытки взлома учетных записей, путем выполнения 
пользователем простого теста, подтверждающего, что он человек, а не программа 
[1].

CAPTCHA является важной мерой безопасности, так как предотвращает автоматические 
атаки, например, массовую регистрацию ботов, и защищает данные пользователя. 
Современные системы CAPTCHA используют не только текст, но и изображения, аудио, 
поведенческие анализы и другие инновационные подходы, чтобы сделать тесты удобными 
для людей, но сложными для программ.

На сегодняшний день наиболее распространенные виды CAPTCHA включают:

\begin{enumerate}
    \item reCAPTCHA -- разработанная Google система, которая предлагает тесты 
    на основе распознавания объектов, анализа поведения или текстовых символов.
    \item hCAPTCHA -- альтернатива reCAPTCHA, фокусирующаяся на защите 
    конфиденциальности пользователей.
    \item Capy -- система CAPTCHA, предлагающая пользователю головоломки, 
    например, сборку изображения или взаимодействие с элементами интерфейса [2].
\end{enumerate}

\section{reCAPTCHA}

reCAPTCHA -- система защиты от автоматизированных действий, разработанная Google, 
которая помогает различать человека и бота. Она объединяет несколько подходов, 
делая проверку удобной для пользователей, но сложной для автоматических систем 
[3].

reCAPTCHA включает в себя следующие версии:

\begin{enumerate}
    \item reCAPTCHA v1 (устарела в 2018 году):
    \begin{enumerate}
        \item пользователи вводили текст, состоящий из искаженных слов, 
        отображаемых на изображении;
        \item использовала слова из книг и документов, которые не могли быть 
        распознаны OCR.
    \end{enumerate}
    \item reCAPTCHA v2:
    \begin{enumerate}
        \item клик по флажку: пользователи подтверждают, что они не роботы, 
        нажимая на флажок «Я не робот»;
        \item выбор объектов на изображениях: пользователи идентифицируют 
        заданные объекты на сетке из картинок;
        \item аудио CAPTCHA: для пользователей с ограничениями зрения, 
        предлагается прослушать запись и ввести услышанные символы.
    \end{enumerate}
    \item reCAPTCHA v3:
    \begin{enumerate}
        \item полностью работает в фоновом режиме, анализируя поведение 
        пользователя на странице;
        \item не требует явных действий, если пользователь считается 
        низкорискованным [4].
    \end{enumerate}
\end{enumerate}

\section{hCAPTCHA}

hCAPTCHA -- это альтернативная система CAPTCHA, разработанная для защиты сайтов 
от ботов и спама, при этом уделяющая особое внимание конфиденциальности 
пользователей. Она стала популярной благодаря своей гибкости и ориентации на 
защиту данных [5].  

Основные особенности hCAPTCHA:

\begin{enumerate}
    \item конфиденциальность:
    \begin{enumerate}
        \item в отличие от reCAPTCHA, hCAPTCHA не собирает данные о 
        пользователях для рекламных целей, что делает ее привлекательной с точки 
        зрения соблюдения конфиденциальности.
    \end{enumerate}
    \item простота интеграции:
    \begin{enumerate}
        \item легко интегрируется с web-сайтами через API;
        \item совместима с большинством популярных платформ, таких как WordPress, 
        и может быть настроена для разных типов взаимодействия.
    \end{enumerate}
    \item модели монетизации:
    \begin{enumerate}
        \item владельцы сайтов могут зарабатывать, разрешая hCAPTCHA 
        использовать проверочные задачи, связанные с машинным обучением, 
        например, разметку данных.
    \end{enumerate}
\end{enumerate}

Виды взаимодействия с пользователями:

\begin{enumerate}
    \item графическая CAPTCHA: выбор изображений, соответствующих запросу;
    \item текстовая CAPTCHA: ввод символов (редко используется);
    \item аудио CAPTCHA: для пользователей с ограниченными возможностями, 
    предлагается прослушать и ввести услышанные символы;
    \item клик CAPTCHA: нажатие на флажок «Я не робот» (для низкорискованных 
    пользователей).
\end{enumerate}

\section{Capy}

Capy CAPTCHA -- это инновационная система CAPTCHA, разработанная с акцентом на 
удобство для пользователей и адаптацию к современным web-средам. Она предлагает 
интерактивные методы проверки, направленные на минимизацию раздражения 
пользователей при сохранении высокого уровня защиты от ботов [6].

Основные особенности Capy CAPTCHA:

\begin{enumerate}
    \item интерактивность:
    \begin{enumerate}
        \item Capy использует методы проверки, которые требуют не просто ввода 
        текста или выбора картинок, а выполнения задач, таких как перемещение 
        объектов;
        \item простые задачи делают процесс проверки менее раздражающим и более 
        интуитивным;
    \end{enumerate}
    \item гибкость настройки:
    \begin{enumerate}
        \item система может быть адаптирована под конкретные нужды сайта, 
        включая выбор сложности задач и дизайн интерфейса.
    \end{enumerate}
    \item доступность:
    \begin{enumerate}
        \item подходит для пользователей с различными потребностями, включая 
        мобильные устройства.
    \end{enumerate}
\end{enumerate}

Виды взаимодействия с пользователями:

\begin{enumerate}
    \item головоломки (Puzzle CAPTCHA): сборка пазла с перемещением недостающих 
    элементов в нужное место;
    \item тесты на логику и распознавание: выбор нужного объекта или 
    логического варианта из предложенных;
    \item текстовая CAPTCHA (редко используется).
\end{enumerate}

Capy CAPTCHA используется на сайтах, где важны как защита от ботов, так и 
положительный пользовательский опыт. Особенно популярна в проектах с высоким 
акцентом на дизайн и пользовательское взаимодействие.
