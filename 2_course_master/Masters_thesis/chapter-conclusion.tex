\chapter*{ЗАКЛЮЧЕНИЕ}
\addcontentsline{toc}{chapter}{ЗАКЛЮЧЕНИЕ}

В ходе выполнения данной работы была рассмотрена проблема автоматизации 
распознавания CAPTCHA различных форматов с применением современных нейросетевых и 
программных инструментов. Актуальность исследования обусловлена постоянным 
усложнением CAPTCHA-систем и параллельным развитием технологий машинного 
обучения, позволяющих преодолевать подобные механизмы защиты.

В рамках исследования поставленная цель: разработка и анализ решений для 
автоматического распознавания текстовых, графических и аудио CAPTCHA -- была 
достигнута.

По результатам работы были решены следующие задачи:

\begin{enumerate}
    \item проведён обзор форматов CAPTCHA и существующих методов защиты от 
    автоматических атак;
    \item реализована система для распознавания текстовых CAPTCHA на основе 
    нейросетевого подхода, обеспечивающего устойчивость к искажениям и фоновому 
    шуму;
    \item создано решение для графических CAPTCHA с использованием модели YOLO, 
    адаптированной для распознавания объектов на изображениях;
    \item реализован подход к решению CAPTCHA в аудиоформате с использованием 
    облачного API распознавания речи, обладающего высокой точностью в условиях 
    фоновых шумов;
    \item проведено тестирование всех компонентов системы в условиях, 
    приближенных к реальным, с подтверждением их корректной и стабильной работы.
\end{enumerate}

Полученные результаты демонстрируют возможность эффективного распознавания 
различных типов CAPTCHA при помощи специализированных моделей и сервисов. 
Решения, основанные на применении нейросетей и облачных технологий, показали 
высокую точность и адаптивность к искажениям в форматах защиты.

Перспективы дальнейших исследований включают:

\begin{enumerate}
    \item расширение набора поддерживаемых типов CAPTCHA, включая более сложные 
    динамические и мультимодальные варианты;
    \item оптимизацию времени обработки и точности распознавания;
    \item исследование механизмов защиты CAPTCHA, устойчивых к современным 
    методам автоматического анализа.
\end{enumerate}

Таким образом, предложенный подход демонстрирует практическую применимость 
современных инструментов машинного обучения и компьютерного зрения в задачах 
анализа и распознавания CAPTCHA, а также может служить основой для дальнейших 
разработок в области тестирования надёжности и устойчивости защитных механизмов 
на web-ресурсах.
