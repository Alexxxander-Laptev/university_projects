\chapter*{ВВЕДЕНИЕ}
\addcontentsline{toc}{chapter}{ВВЕДЕНИЕ}

С развитием цифровых технологий и ростом интернет-активности существенно возросла 
потребность в защите web-ресурсов от автоматизированного взаимодействия. Одним 
из ключевых инструментов такой защиты являются системы CAPTCHA (Completely 
Automated Public Turing test to tell Computers and Humans Apart), задача которых 
-- отличить действия человека от автоматического скрипта~\cite{captchawiki}. 
CAPTCHA применяется для предотвращения спама, злоупотреблений при регистрации, 
массовых запросов к сервисам и подобных форм мошеннической активности.

Современные системы CAPTCHA предлагают множество форматов: текстовые (с 
искажённым символьным изображением), графические (выбор изображений по 
заданному критерию), а также аудио (воспроизведение и распознавание голосовой 
записи в условиях шумов). Одновременно с этим появляются возможности для их 
автоматического распознавания, в том числе с использованием методов 
машинного обучения и нейросетевых архитектур.

Актуальность данной работы обусловлена как возрастающей сложностью 
CAPTCHA-систем, так и развитием инструментов, позволяющих преодолевать защитные 
механизмы web-ресурсов. Анализ эффективности и разработка подходов для 
автоматизированного решения CAPTCHA могут применяться не только с точки зрения 
изучения устойчивости самих систем, но и в рамках исследования прикладного 
применения нейросетевых моделей в задачах распознавания информации в условиях 
ограничений~\cite{captchatrouble2}.

Целью данной работы является разработка и анализ комплексного решения к 
автоматизации решения CAPTCHA в различных форматах с использованием современных 
нейросетевых инструментов и API для распознавания.

Для достижения поставленной цели были сформулированы следующие задачи:

\begin{enumerate}
    \item провести обзор существующих форматов CAPTCHA и методов их защиты;
    \item разработать систему автоматического распознавания текстовых CAPTCHA с 
    искажениями;
    \item реализовать подход к решению графических CAPTCHA на основе методов 
    компьютерного зрения и нейросетевых моделей;
    \item построить решение для аудио CAPTCHA с использованием средств 
    автоматического распознавания речи;
    \item протестировать реализованные решения в реальных условиях, оценить 
    точность распознавания и стабильность работы.
\end{enumerate}
