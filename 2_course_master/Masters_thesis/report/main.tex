\documentclass{altsu-report}

\begin{document}

\section*{Цель и задачи работы}

Одной из наиболее сложных форм CAPTCHA являются изображения, содержащие множество объектов с размытыми контурами, шумами и низким разрешением, что затрудняет автоматическое распознавание.

В процессе автоматизированного тестирования web-приложений возникает необходимость обхода подобных CAPTCHA, что требует разработки устойчивых и точных методов распознавания визуального контента.

Целью данной работы является разработка и обучение нейронной сети с поддержкой сегментации, способной автоматически распознавать объекты на изображениях CAPTCHA и выполнять задания, формируемые системой защиты.  

Для достижения поставленной цели необходимо решить следующие задачи:
\begin{enumerate}
    \item проанализировать типы CAPTCHA, применяемых на web-ресурсах;
    \item выбрать подходящую архитектуру нейронной сети, обеспечивающую высокую скорость и точность;
    \item собрать и разметить датасет реальных CAPTCHA с изображениями объектов;
    \item провести предварительную обработку изображений и формирование структуры датасета;
    \item обучить выбранную модель на собранных данных;
    \item разработать скрипт для автоматизированного прохождения CAPTCHA с использованием обученной модели;
    \item протестировать модель в реальных условиях и оценить её эффективность.
\end{enumerate}

\section*{Выбор модели нейронной сети для обучения}

CAPTCHA в формате изображений широко используется для защиты ресурсов от автоматизированных ботов и может быть реализована несколькими способами. Как правило, такие CAPTCHA направлены на проверку способности пользователя распознавать и интерпретировать объекты на изображении. Наиболее распространены два варианта реализации (оба варианта реализации проиллюстрированы на слайде:

\begin{enumerate}
    \item цельное изображение, содержащее несколько объектов, частично размытых или искажённых, при этом изображение разбито на сетку 3×3 или 4×4. Пользователю предлагается выбрать ячейки, содержащие объекты определённого класса (например, автобусы или светофоры);
    \item составное изображение, сформированное из 9 или 12 отдельных фрагментов (изображений), каждый из которых представляет собой независимое изображение -- зачастую низкого качества, с наложением артефактов или шумов. Задача пользователя -- выбрать те изображения, где присутствует нужный объект.
\end{enumerate}

Такие CAPTCHA требуют от системы автоматического анализа способности как к глобальному восприятию изображения, так и к локальной интерпретации его фрагментов. Соответственно, модель, предназначенная для решения данной задачи, должна поддерживать:

\begin{enumerate}
    \item классификацию объектов на уровне отдельных изображений (для CAPTCHA, основанных на отдельных картинках в сетке);
    \item локализацию и сегментацию объектов с высокой точностью, чтобы корректно определить границы объектов в пределах ячеек, особенно в случаях, когда объект может частично заходить за границу между ячейками.
\end{enumerate}

\textbf{Переключение слайда}

Для решения этих задач были рассмотрены следующие современные архитектуры нейронных сетей:

\begin{enumerate}
    \item YOLO;
    \item Faster R-CNN;
    \item DETR.
\end{enumerate}

Среди этих архитектур было принято решение использовать YOLOv8 по причинам, указанным на слайде:

Кроме того, модель YOLOv8 была успешно протестирована в задачах, близких по структуре к CAPTCHA: детекции дорожных знаков, транспортных средств, пешеходов и других объектов в сложных условиях съёмки, что подтверждает её универсальность и применимость к рассматриваемой задаче.

Таким образом, YOLOv8 является наиболее сбалансированным выбором, обеспечивающим как точную классификацию, так и локализацию объектов в условиях ограниченных ресурсов и с возможностью адаптации под специфику визуальных CAPTCHA.

\section*{Парсинг CAPTCHA для создания датасета}

Для обеспечения высокой точности в задаче автоматического решения CAPTCHA необходимо подготовить собственный набор данных, приближённый к реальным условиям использования. Наиболее эффективным методом является автоматизированный парсинг изображений CAPTCHA, представленных на веб-сайтах, использующих визуальные CAPTCHA-решения, такие как Google reCAPTCHA v2.

Использование реальных CAPTCHA, собранных в автоматическом режиме, имеет ряд преимуществ по сравнению с синтетической генерацией данных:

\begin{enumerate}
    \item изображения содержат разнообразные сцены, освещение, углы обзора и уровни шума, что положительно влияет на способность модели к обобщению;
    \item присутствует большое количество уникальных объектов на фоне, в том числе в частично перекрытых и смазанных вариантах;
    \item отсутствует необходимость в ручной генерации изображений и создании дополнительных искажений для повышения реалистичности;
    \item возможно извлекать текстовые инструкции к CAPTCHA, что позволяет соотносить каждое изображение с требуемым классом.
\end{enumerate}

Для парсинга CAPTCHA был реализован автоматизированный сценарий взаимодействия с браузером с использованием библиотеки Selenium, блок-сехма которого показана на данном слайде.

\section*{Предобработка изображений датасета}

После получения достаточного количества изображений для составления датасета необходимо провести их предварительную обработку и разметку. Это один из самых важных этапов работы, поскольку от качества разметки напрямую зависит точность и эффективность последующей работы модели.

Для создания меток используется инструмент CVAT -- многофункциональное веб-приложение с поддержкой аннотации объектов с помощью полигонов, прямоугольников и других форм. CVAT позволяет экспортировать разметку напрямую в формат, совместимый с YOLO.

Поскольку CAPTCHA-изображения часто содержат объекты с нечёткими контурами, наложением и визуальными искажениями, особенно важно использовать ручную точную разметку, а не ограничиваться автоматическими методами. Выделение объектов должно проводиться как можно точнее, с учётом геометрии контуров. На слайде представлен пример изображения с размеченными объектами.

\section*{Обучение модели на датасете}

В качестве основной архитектуры была выбрана модель YOLOv8m-seg, поддерживающая сегментацию объектов.

Преимущества YOLOv8m-seg заключаются в следующем:

\begin{enumerate}
    \item наличие встроенной поддержки сегментации объектов;
    \item возможность использования предобученных весов;
    \item высокая скорость инференса по сравнению с другими моделями сегментации;
    \item встроенные средства аугментации;
    \item удобный интерфейс через библиотеку ultralytics, позволяющий быстро запускать обучение, логировать метрики и визуализировать результаты;
    \item полная совместимость с аннотациями в формате YOLO, полученными из CVAT.
\end{enumerate}

Обучение проводилось на 35 эпохах при размере изображений 640×640 пикселей и размере батча 8. Использование предобученных весов позволило достичь стабильного снижения функции потерь с первых эпох, а встроенные механизмы аугментации способствовали улучшению обобщающей способности модели.

Результаты обучения отслеживались по ключевым метрикам (IoU, Precision, Recall, Loss), которые визуализировались автоматически. Примеры графиков с результатами обучения приведены на данных слайдах.

\section*{Тестирование модели}

После завершения обучения модель была протестирована на реальных CAPTCHA, собранных с помощью автоматического парсера, реализованного на базе библиотеки Selenium.

Тестирование было организовано в виде цикла, позволяющего автоматически проходить CAPTCHA до тех пор, пока не будет достигнут положительный результат. Это позволило зафиксировать частоту ошибок модели и определить случаи, в которых требуются дообучение или оптимизация.

Рабочий процесс тестирования и взаимодействия модели с CAPTCHA представлен на блок-схеме.

\section*{Заключение}

В рамках данной работы была реализована система автоматического распознавания и прохождения CAPTCHA с изображениями, основанная на использовании нейросетевой модели YOLOv8 с поддержкой сегментации и были решены все поставленныые задачи.
    
\end{document}
