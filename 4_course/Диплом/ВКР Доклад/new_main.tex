\documentclass[12pt,a4paper]{article}
\linespread{1,15}
\usepackage[utf8]{inputenc}
\usepackage[T2A]{fontenc}
\usepackage[russian]{babel}
\usepackage{amsmath}
\usepackage{amsfonts}
\usepackage{amssymb}
\usepackage{graphicx}
\usepackage{listings}
\usepackage[title,titletoc]{appendix}
\usepackage[left=30mm, top=20mm, right=15mm, bottom=20mm, nohead, footskip=10mm]{geometry}
\usepackage{indentfirst}

\begin{document}

% \maketitle

\section{Введение}

В самом начале я хотел бы немного рассказать о том, что такое DnD.

DnD -- это настольная ролевая игра в жанре фэнтези. Действующими лицами в игре являются Мастер -- исполняет роль ведущего и игроки -- играют от лица своего персонажа. Для игры в эту настольную ролевую игру каждый игрок должен создать своего персонажа с определенными навыками и характеристиками. Основной целью игры Dungeons \& Dragons является описание персонажа и его приключений. Для создания персонажа могут быть задействованы игральные кости и Основные правила, а также личные предпочтения игроков.

Основным средством для описания персонажа  служит лист персонажа, который представляет собой сгруппированный в удобном виде набор характеристик персонажа. К таким характеристикам относятся следующие:

\begin{enumerate}
    \item Основные атрибуты, присущие каждому персонажу: Сила, Телосложение, Ловкость, Интеллект, Мудрость, Харизма;

    \item Особые умения, которые индивидуализируют персонажа: способность к убеждению или расследованию;

    \item Определенные действия, такие как атака оружием или произнесение заклинаний;

    \item Языки, на которых говорит персонаж, или инструменты, которыми он умеет пользоваться.
\end{enumerate}

С точки зрения данной работы DnD расценивается именно как набор правил, по которым может осуществляться конструирование персонажа.

\section{Правила для генерации базовых характеристик персонажа}

Каждый персонаж в игре обладает следующими характеристиками: Сила, Телосложение, Ловкость, Интеллект, Мудрость и Харизма. Значение каждой из этих характеристик выражается определенным количеством очков, которое выражается числом.

Процесс генерации может происходить различными способами, но в основном используют эти:

\begin{enumerate}
    \item 3d6 -- классический способ, генерация осуществляется броском 3d6 (трёхкратным броском шестигранного кубика). Такой метод дает распределение очков характеристики от 3 до 18.

    \item 4d6 -- один из альтернативных способов, генерация осуществляется броском 4d6 (четырехкратным броском игральной кости), а минимальное значение среди бросков отбрасывается. На данный момент такой метод применяется чаще остальных, поскольку такое распределение также дает разброс от 3 до 18, но вероятность получения более высокого итогового значения возрастает.

    \item Персонаж начинает со всеми характеристикам равными 8. У игрока есть 27 очков, которые он может свободно распределить, при этом 14 и 15 значение характеристики стоят по 2 очка каждое.

    \item В D\&D 5e основной метод определения характеристик -- выбор из набора стандартных значений: 15, 14, 13, 12, 10, 8.
\end{enumerate}

\section{Выбор расы и класса персонажа}

Помимо базовых характеристик персонаж является фэнтезийным существом, а значит обладает расовой и классовой принадлежностью. Эти характеристики позволяют описать персонажа и его предпочтительный род деятельности. Представителям определенных рас зачастую больше подходят определенные классы, но в целом выбор данных параметров -- выбор каждого игрока. В некоторых случаях от выбора расы и класса могут изменяться характеристики персонажа.

Самые распространенные расы в игре:

Самые распространенные классы в игре:

Помимо данных характеристик персонаж может обладать каким-то снаряжением или иметь предрасположенность к магии и заклинаниям.

\section{Актуальность}

Все вышеописанные правила являются лишь базовым набором и видно, что даже на знакомство с ними для создания своего персонажа игроку, в особенности начинающему, нужно потратить немало сил и времени.

Поэтому существовала необходимость для облегчения и упрощения этого процесса. на сегодняшний день для D\&D разработано огромное количество генераторов персонажей от энтузиастов и крупных компаний, которые облегчают игрокам процесс игры и генерации своего персонажа.

Но все эти генераторы и конструкторы персонажей представляют собой Web-приложения или мобильные приложения, которым для работы нужен выход в Интернет.

\section{Цель и задачи работы}

Такая возможность есть не везде и не всегда, поэтому я поставил себе цель создать портативное устройство для генерации персонажа D\&D, которое будет приспособлено для повседневного использования в любом месте без подключения к Интернету.

Для достижения поставленной цели требуется выполнить следующие задачи:

\begin{enumerate}
    \item Разработка схемы электрической принципиальной устройства и подбор компонентов, необходимых для его сборки;

    \item Проектирование и трассировка печатной платы, установка на нее всех компонентов схемы;

    \item Сборка и корпусирование готового устройства;

    \item Разработка программного обеспечения для помощи игрокам на любом этапе игры, согласно правилам D\&D5;

    \item Отладка и тестирование работоспособности программной и аппаратной части устройства.
\end{enumerate}

\section{Примеры существующих конструкторов}

На данном слайде представлены примеры конструкторов и генераторов персонажей, которые пользуются наибольшей популярностью среди игроков и обладают наибольшим функционалом.

\begin{enumerate}
    \item D\&D Beyond

    \item AideDD

    \item Хранилище мастера подземелий

    \item Быстрый конструктор персонажей

    \item DDNext

    \item Генератор персонажей Леви Блоджетта

    \item Roll20

    \item Dungeon Master’s Vault

    \item 5e Companion

    \item Ninetale

    \item MorePurpleMoreBetter
\end{enumerate}

У каждого из этих генераторов есть свои особенности и фишки, которых нет у других генераторов, но объединяет их то, что это одни из самых настраиваемых приложений для генерации персонажей DnD.

\section{Необходимый функционал}

Для того, чтобы помочь пользователю и избавить его от основной части расчетов генератор должен обладать следующим функционалом:

\begin{enumerate}
    \item Выбор расы персонажа;

    \item Выбор класса персонажа;

    \item Генерация значений 5 базовых характеристик:

    \begin{enumerate}
        \item Сила (Strong --- сокращенно <<Str>>);
        
        \item Телосложение (Constitution --- сокращенно <<Con>>);
        
        \item Ловкость (Dexterity --- сокращенно <<Dex>>);
        
        \item Интеллект (Intelligence --- сокращенно <<Int>>);
        
        \item Мудрость (Wisdom --- сокращенно <<Wis>>);
        
        \item Харизма (Charisma --- сокращенно <<Cha>>).
    \end{enumerate}

    \item Расчет значений для ряда побочных характеристик: класс защиты и хит-поинтов персонажа (на основе базовых характеристик);

    % \item Выбор снаряжения на основе ранее выбранных игроком характеристик;

    % \item Еще некоторые побочные характеристики.
\end{enumerate}

Указанный функционал является минимально необходимым для генерации персонажа с базовыми характеристиками и может быть расширен для более глубокой проработки персонажа, но в то же время позволяет существенно снизить нагрузку на игрока, который создает персонажа

\section{Схемы}



\section{Блоки кода}

\subsection{Блок-схема}

На данном слайде представлена блок-схема алгоритма для выбора расы персонажа. В данной части кода осуществляется работа с дисплеем и кнопками. Есть список, прокрутка по которому осуществляется с помощью кнопок LEFT, UP, RIGHT, DOWN, после нажатие на которые обновляется содержимое дисплея. После того, как игрок определился с расой он может нажать кнопку SELECT, после чего осуществится вызов подпрограммы для выбора класса персонажа.

\subsection{Блок кода для выбора класса}

Данный блок кода по своей сути и принципу работы аналогичен подпрограмме для выбора расы персонажа, единственное отличие в содержимом, которое выводится на дисплей.

\subsection{Блок кода для генерации базовых характеристик персонажа}

В данном блоке происходит генерация шести базовых характеристик, про которые было сказано раньше. Здесь имитируется бросок игральной кости и происходит генерация согласно методу 4d6. Сгенерированные таким образом значения заносятся в массив, откуда потом с ними и происходит взаимодействие.

\section{Заключение}

В заключение я хотел бы сказать о том, что в ходе работы был выполнен ряд задач, который приведен на текущем слайде и в результате работы было создано устройство для генерации персонажей DnD с базовыми характеристиками согласно основным правилам игры.

\end{document}
