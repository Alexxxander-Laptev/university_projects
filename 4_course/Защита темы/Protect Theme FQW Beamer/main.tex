% Пример заготовки для презентации с использованием класса Beamer LaTeX.
% Версия от 09 ноября 2018 года.
\documentclass[12pt,a4paper,mathserif]{beamer}
\usepackage[utf8x]{inputenc}
\usepackage{ucs}
\usepackage[T2A]{fontenc}
\usepackage[english,russian]{babel}
\usepackage{amsmath}
\usepackage{amsfonts}
\usepackage{amssymb}
\usepackage{mathtext}
\usepackage{graphicx}
\usepackage{enumerate}
\usepackage{multirow}
\usepackage{ragged2e}
% Пакет для оформления исходного кода
\usepackage{minted}
\usepackage{adjustbox}
\justifying
\renewcommand{\raggedright}{\leftskip=0pt \rightskip=0pt plus 0cm}
\setbeamertemplate{caption}[numbered]

\usetheme {Madrid}
\usecolortheme [RGB={85, 107, 47}]{structure} %Dark Olive Green

\author[Лаптев А.В.]{{Cтудент 595 группы: Лаптев А. В.}\\
{Научный руководитель: Шмаков И.А.}}
\title[Научно-исследовательская работа]{Проектирование и разработка устройства для генерации персонажа Dangeous \& Dragons}
% \subtitle{Отчет по научно-исследовательской работе}

\begin{document}
\begin{frame}
\maketitle
\end{frame}

\begin{frame}{Цель и задачи работы}
    \setlength{\parindent}{0.5cm}
    Целью работы является проектирование и создание устройства, которое будет предназначено для генерации персонажа DnD, а также разработка программного обеспечение для данного устройства с функционалом, необходимым для помощи игрокам на протяжении всей игры.
\end{frame}

\begin{frame}{Цель и задачи работы}
    \setlength{\parindent}{0.5cm}
    Задачи работы:
    
    \begin{enumerate}
        \item Сравнение параметров различных микроконтроллеров, с целью подобрать оптимальный вариант по соотношению "цена-качество";
    
        \item Разработка принципиальной схемы устройства и подбор компонентов, необходимых для его сборки;
    
        \item Проектирование и трассировка печатной платы, установка на нее всех компонентов схемы;
    
        \item Сборка и корпусирование готового устройства;
    
        \item Разработка программного обеспечения для помощи игрокам на любом этапе игры, согласно правилам DnD5;
    
        \item Отладка и тестирование работоспособности программной и аппаратной части устройства.
    \end{enumerate}
\end{frame}

\begin{frame}{Актуальность выбранной темы}
    \setlength{\parindent}{0.5cm}
    Для DnD разработано большое количество Web-ресурсов, приложений и ботов от энтузиастов и крупных компаний, которые облегчают игрокам процесс игры и генерации своего персонажа.
    
    О масштабах распространения игры можно судить по популярности таких приложений, как DnD Beyond, как среди пользователей, так и среди крупных компаний. В конце мая этого года он был куплен компанией Hasbro, которая занимается его дальнейшей поддержкой. Количество пользователей DnD Beyond насчитывает около 10 млн.
\end{frame}

\begin{frame}{Промежуточные результаты работы}
    \setlength{\parindent}{0.5cm}
    На данный момент, имеется работающий вариант генератора персонажа, собранный на отладочной плате Arduino Uno, с использованием платы расширения с LCD-дисплеем, а также, реализован базовый функционал генератора.
\end{frame}

\begin{frame}{Ожидаемые результаты работы}
    \setlength{\parindent}{0.5cm}
    В ходе выполнения работы планируется выполнить все поставленные задачи и в результате предоставить полноценное работоспособное устройство, а также продемонстрировать работоспособность разработанного программного обеспечения на этом устройстве.
\end{frame}
\end{document}